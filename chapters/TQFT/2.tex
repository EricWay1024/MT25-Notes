\section{Particle Quantum Statistics}

\subsection{Single Particle Path Integral}


Let $\vb x (t) \in \mathbb{R}^D$ be the position of a single particle moving in $\mathbb{R}^D$ at time $t$. Let $\vb x_i$ be its initial position at time $t_i$ and $\vb x_f$ its final position at time $t_f$. Let $\hat{U}$ be a unitary time evolution operator, then the \textbf{propagator} gives the amplitude of starting at $\vb x_i$ and ending at $\vb x_f$, written as $$
\langle \vb x_f | \hat{U} (t_f, t_i) | \vb x_i \rangle
.$$        

Suppose we have an arbitrary wavefunction at $t_i$: $\psi (\vb x_i , t_i) = \langle \vb x_i | \psi (t_i)  \rangle$. Then we can use the propagator to propagate forward $\psi$ to $t_f$: $$
\langle \vb x_f | \psi (t_f) \rangle = \int d \vb x_i \langle \vb x_f | \hat{U} (t_f, t_i) | \vb x_i \rangle \langle \vb x_i | \psi (t_i)  \rangle
.$$  
In other words, the later wavefunction is a superposition of all ways the particle could have started at every $\vb x_i$, each weighted by the propagator from $\vb x_i$ to $\vb x_f$. 


The propagator has two properties:
\begin{itemize}
    \item It is unitary, so no amplitude is lost along the way.
    \item It respects path composition, so propagating from $t_i$ to $t_m$ and then from $t_m $ to $t_f$, is the same as propagating from $t_i$ to $t_f$. 
\end{itemize}

Feynman says that the propagator can be rewritten as 
$$
\langle \vb x_f | \hat{U} (t_f, t_i) | \vb x_i \rangle = \mathcal{N} \sum e^{i S[\vb x(t)] / \hbar }
,$$
where $\mathcal{N}$ is a constant,  the sum is over all paths $\vb x(t)$ (in the $(D+1)$-dimensional spacetime) from $(\vb x_i, t_i)$ to $(\vb x_f, t_f)$, and $S[\vb x(t)]$ is the classical action of the path (which is the integral of the Lagrangian).

For simplicity we will assume the Plunk constant $\hbar = 1$. 

The sum is dominant by the paths where the exponent oscillates the least, i.e. the paths with the least action. 

\subsection{Two Identical Particles}


If we have two identical particles, then the two states $| \vb x_1, \vb x_2 \rangle$  and $| \vb x_2, \vb x_1 \rangle$ are equivalent. We call the space of all states the \textbf{configuration space} $\mathcal{C}$. 



We call the non-exchange path TYPE $+1$ and the exchange path TYPE $-1$. Clearly their composition obeys the group operations of $\mathbb{Z}_2$. 


The path integral is now written as follows:

\begin{equation*}
\label{eq:3.4}
\langle x_{1f}\,x_{2f}\,|\,\hat U(t_f,t_i)\,|\,x_{1i}\,x_{2i}\rangle
= \mathcal{N}\sum_{\text{paths } i\to f} e^{i S[\text{path}]/\hbar}
= \mathcal{N}\!\left(
\sum_{\substack{\text{TYPE }+1\ \text{paths}\\ i\to f}} e^{i S[\text{path}]/\hbar}
+ \sum_{\substack{\text{TYPE }-1\ \text{paths}\\ i\to f}} e^{i S[\text{path}]/\hbar}
\right).
\end{equation*}

% This second line is simply a rewriting of the first having broken the sum into
% the two different classes of paths.


However, if we work with a different type of particles, we may have the following case:

\begin{equation*}
\label{eq:3.5}
\langle x_{1f}\,x_{2f}\,|\,\hat U(t_f,t_i)\,|\,x_{1i}\,x_{2i}\rangle
= \mathcal{N}\!\left(
\sum_{\substack{\text{TYPE }+1\ \text{paths}\\ i\to f}} e^{i S[\text{path}]/\hbar}
- \sum_{\substack{\text{TYPE }-1\ \text{paths}\\ i\to f}} e^{i S[\text{path}]/\hbar}
\right).
\end{equation*}

This case holds mathematically because it respects path composition. 


\subsection{Many Identical Particles}

The configuration space can be described as follows:
$$
\mathcal{C} = [(\mathbb{R}^D)^N - \Delta] / \sim 
,$$
where $\Delta$ represents \textbf{coincidences} where two particles are at the same position (or with a distance less than $\varepsilon$), and $\sim $ is the equivalence relation induced by the permutation of coordinates.

\subsubsection{2+1 D}

The paths in $2+1$ D can be classified using the braid group $B_N$. It is generated by $\sigma_1, \ldots , \sigma_{N-1}$, where $\sigma_i$  exchanges the $i$-th and the $(i+1)$-th strand while giving an overcrossing. 


A braid invariant is given by the \textbf{winding number}, which is the number of overcrossings minus the number of undercrossings. This is also the number of $\sigma_i$'s minus the number of $\sigma_i ^{-1}$'s in the braid word.

\subsubsection{3+1 D}

There is no knot in the 4 dimensional spacetime, so all the paths are simply classified by the symmetric group $S_N$. 


\subsubsection{Building a path integral}

Suppose the paths are classified by group $G$. We have
 $$
\langle \left\{ \vb x \right\} _f  | \hat{U} (t_f , t_i) | \left\{ \vb x \right\} _i  \rangle = \mathcal{N} \sum_{g \in  G} \rho(g) \sum_{\text{paths with type } g} e^{i S[\text{path}] / \hbar}
.$$

Here $\rho$ is a unitary representation of $G$. Adding $\rho(g)$ to the summation  is in line with our previous observation that adding a minus sign to TYPE -1 paths is a consistent move that respects path composition; and a different $\rho$ would  in fact correspond to a different type of particles. 


\subsection{Abelian Examples}


Now suppose $\rho$ is a one-dimensional representation of $G$; hence it is abelian. 

\subsubsection{3+1 D}

\begin{lemma}
  There are only two possible one-dimensional representations of $S_N$.  
\end{lemma}

\begin{proof}
    Clearly $\rho$ can only send a transposition to either $+1$ or $-1$.   
  Since $S_N$ is generated by transpositions, it suffices to show that if $\rho$ sends any one transposition to $+1$ resp.  $-1$, then $\rho$ sends all transpositions to $+1$ resp. $-1$. But we also know that any two transpositions are conjugate, and since $\rho$ is abelian, the result easily follows.      
\end{proof}

These two reps correspond to two different particles:

\begin{itemize}
    \item The trivial representation $\rho (g) = 1$ for all $g$. This corresponds to \textbf{bosons}.
    \item The alternating (sign) representation $\rho(g) = \operatorname{sgn} (g) = \pm 1$. This corresponds to \textbf{fermions}.    
\end{itemize}

\subsubsection{2+1 D}


We have a family of one-dimensional representations of $B_N$, parametrized by $\theta \in \mathbb{R} \pmod {2 \pi}$, given by $$
\rho(g) = e^{i \theta W(g)}
,$$
where $W(g)$ is the winding number of the braid $g$. In other words,
a clockwise resp. anticlockwise exchange accumulates a phase of $e^{i \theta}$ resp. $e^{- i \theta}$. 



\begin{itemize}
    \item If $\theta = 0$ then we recover \textbf{bosons}.
    \item If $\theta = \pi$ then we recover \textbf{fermions}.
    \item For any other value of $\theta$ we have \textbf{(abelian) anyons} or \textbf{fractional statistics}.   
\end{itemize}

\subsection{Non-abelian case}

\TODO







\section{Aharonov--Bohm Effect and Charge-Flux Composites}



\subsection{Aharonov--Bohm Effect}

I don't really know. But it seems that the effect says if there is a magnetic field between the two paths of the double-silt interference experiment, the amplitude (or the measured interference pattern) will be affected, even if the particle does not experience the (constant) field either directly or via the Faraday effect. Via some computation, what really happens is that the presence of this field gives a difference in accumulated phases the two paths. The phase difference is given by $$
\exp \left(  \frac{iq}{\hbar} \Phi_{\text{enclosed}}  \right)
,$$
where $\Phi_{\text{enclosed}}$ is the flux enclosed by the loop (path 1 - path 2). In other words, the phase shift is what we get when we move a charge (particle) around a flux.  


That seems to be all that matters.

\begin{itemize}
    \item If $\Phi \in \Phi_0 \mathbb{Z} $, where $\Phi_0 = 2 \pi \hbar / q$ is the elementary flux quantum, then the phase shift is in $2 \pi \mathbb{Z}$ and thus is equivalent to $0$.     
    \item We would get the same phase shift if we were to move flux around a charge. 
\end{itemize}


\subsection{Anyons as Charge-Flux Composites}


Now for some magical reasons, we model a particle as a charge-flux composite object, denoted as $(q, \Phi)$. 

\begin{proposition}
  If particle one $(q_1, \Phi_1)$  goes around the flux of particle two $(q_2, \Phi_2)$ , then the phase change is $e^{ i (q_1 \Phi_2 + q_2 \Phi_1) / \hbar}$. 
\end{proposition}

\begin{proof}
  This is because $q_1$ going around $\Phi_2$  gives a phase shift of $q_1 \Phi_2$, but at the same time $\Phi_1$ also goes around $q_2$ which gives $q_2 \Phi_1$.     
\end{proof}



\begin{proposition}
  If the two particles both have $(q, \Phi)$, then the phase for exchanging them is $e^{i q \Phi / \hbar}$.  
\end{proposition}


\begin{proof}
  This is because if we exchange them twice, that is equivalent to particle one going around particle two once.
\end{proof}


This indicates that these particles are $\theta$-anyons, where ${\theta} = q \Phi / \hbar$. 


\subsubsection{Spin of an anyon}

\begin{proposition}
  If we rotate the anyon around its axis, we gain a new phase of $e^{i \theta}$. 
\end{proposition}

\begin{proof}
  We can put the charge and the flux at slightly different positions; then the rotation is just moving the charge and the flux around each other, which gives us the result. 
\end{proof}

Note that the phase obtained by exchanging two identical particles is the same as the phase obtained by rotating one around its own axis. This matches the spin-statistics theorem. See Fig. 2.7 (p.10) of the book.


\subsubsection{Fusion of Anyons}

\begin{proposition}
  If we put two particles with $(q_1, \Phi_1)$ and $(q_1, \Phi_2)$   together, then we have a $(q_1+ q_2, \Phi_1+ \Phi_2)$ particle. 
\end{proposition}

\begin{proof}
  We can just pretend that we move far away enough so that they seem they are together.
\end{proof}



\subsubsection{Anti-Anyons}

The \textbf{anti-anyon} of the $(q, \Phi)$ anyon is just the particle with $(-q, -\Phi)$. If we fuse an anyon with its anti-anyon, we get the \textbf{vacuum particle}, $(0,0)$.


\subsection{Torus}

Now let us suppose our space is a torus (and as usual we have the one-dimensional time).
Denote the two nontrivial cycles around the torus as $C_1$ and $C_2$.   

Define two operations $T_1, T_2$ as follows. $T_i$ is the operator that creates a particle-antiparticle pair, moves the two in opposite directions  around the $C_1$ cycle until they meet on the opposite side of the torus and reannihilate. 

We also define $T_i ^{-1}$ as the time-reversed process of $T_i$.


\begin{proposition}
  Both $T_i$ commute with the Hamiltonian $H$; in other words, they preserve the energy. 
\end{proposition}
\begin{proof}
  They just do.
\end{proof}



\begin{proposition}
  $T_1$ does not commute with $T_2$.  
\end{proposition}

\begin{proof}
  It suffices to show that the operator $T_2  ^{-1} T_1 ^{-1} T_2 T_1$ is not trivial. See Fig. 4.9 of the book. The process is equivalent to one particle wrapping around another which gives a phase of $e^{-2 i \theta}$.   
\end{proof}


% \TODO{figure out what ground states mean}

\begin{proposition}
  If $| \alpha \rangle$ is an eigenstate of $T_1$ in the ground state space with $T_1 | \alpha \rangle = e^{i \alpha} | \alpha \rangle$ (which exists since $T_1$ commutes with Hamiltonian), then $T_2 | \alpha \rangle = | \alpha + 2 \theta \rangle$  is another ground state and is another eigenstate of $T_1$ with eigenvalue $e^{i (2 \theta + \alpha)}$.
\end{proposition}


\begin{proof}
  $T_2 |  \alpha \rangle$ must also be in the ground state space since $T_2$ commutes with the Hamiltonian.  Now $$
  T_1 (T_2 | \alpha \rangle) = e^{2 i \theta} T_2 T_1 |  \alpha \rangle = e^{i(2 \theta + \alpha)} (T_2 |  \alpha \rangle)
  .$$
  Therefore $T_2 |  \alpha \rangle$ is also an eigenstate of $T_1$, which must also be in the ground state space.   
\end{proof}

\begin{corollary}
  If $\theta = \pi p / m$ for coprime integers $p, m$, then there are $m$ independent ground states.
\end{corollary}


\begin{proof}
  This is easy, since we can apply the above procedure $m$ times before we get the phase $\alpha + 2 \pi$ which is equivalent to $\alpha$.   
\end{proof}


