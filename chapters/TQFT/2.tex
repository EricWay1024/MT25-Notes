\section{Particle Quantum Statistics}

\subsection{Single Particle Path Integral}


Let $\vb x (t) \in \mathbb{R}^D$ be the position of a single particle moving in $\mathbb{R}^D$ at time $t$. Let $\vb x_i$ be its initial position at time $t_i$ and $\vb x_f$ its final position at time $t_f$. Let $\hat{U}$ be a unitary time evolution operator, then the \textbf{propagator} gives the amplitude of starting at $\vb x_i$ and ending at $\vb x_f$, written as $$
\langle \vb x_f | \hat{U} (t_f, t_i) | \vb x_i \rangle
.$$        

Suppose we have an arbitrary wavefunction at $t_i$: $\psi (\vb x_i , t_i) = \langle \vb x_i | \psi (t_i)  \rangle$. Then we can use the propagator to propagate forward $\psi$ to $t_f$: $$
\langle \vb x_f | \psi (t_f) \rangle = \int d \vb x_i \langle \vb x_f | \hat{U} (t_f, t_i) | \vb x_i \rangle \langle \vb x_i | \psi (t_i)  \rangle
.$$  
In other words, the later wavefunction is a superposition of all ways the particle could have started at every $\vb x_i$, each weighted by the propagator from $\vb x_i$ to $\vb x_f$. 


The propagator has two properties:
\begin{itemize}
    \item It is unitary, so no amplitude is lost along the way.
    \item It respects path composition, so propagating from $t_i$ to $t_m$ and then from $t_m $ to $t_f$, is the same as propagating from $t_i$ to $t_f$. 
\end{itemize}

Feynman says that the propagator can be rewritten as 
$$
\langle \vb x_f | \hat{U} (t_f, t_i) | \vb x_i \rangle = \mathcal{N} \sum e^{i S[\vb x(t)] / \hbar }
,$$
where $\mathcal{N}$ is a constant,  the sum is over all paths $\vb x(t)$ (in the $(D+1)$-dimensional spacetime) from $(\vb x_i, t_i)$ to $(\vb x_f, t_f)$, and $S[\vb x(t)]$ is the classical action of the path (which is the integral of the Lagrangian).

For simplicity we will assume the Plunk constant $\hbar = 1$. 

The sum is dominant by the paths where the exponent oscillates the least, i.e. the paths with the least action. 

\subsection{Two Identical Particles}


If we have two identical particles, then the two states $| \vb x_1, \vb x_2 \rangle$  and $| \vb x_2, \vb x_1 \rangle$ are equivalent. We call the space of all states the \textbf{configuration space} $\mathcal{C}$. 



We call the non-exchange path TYPE $+1$ and the exchange path TYPE $-1$. Clearly their composition obeys the group operations of $\mathbb{Z}_2$. 


The path integral is now written as follows:

\begin{equation*}
\label{eq:3.4}
\langle x_{1f}\,x_{2f}\,|\,\hat U(t_f,t_i)\,|\,x_{1i}\,x_{2i}\rangle
= \mathcal{N}\sum_{\text{paths } i\to f} e^{i S[\text{path}]/\hbar}
= \mathcal{N}\!\left(
\sum_{\substack{\text{TYPE }+1\ \text{paths}\\ i\to f}} e^{i S[\text{path}]/\hbar}
+ \sum_{\substack{\text{TYPE }-1\ \text{paths}\\ i\to f}} e^{i S[\text{path}]/\hbar}
\right).
\end{equation*}

% This second line is simply a rewriting of the first having broken the sum into
% the two different classes of paths.


However, if we work we a different type of particles, we may have the following case:

\begin{equation*}
\label{eq:3.5}
\langle x_{1f}\,x_{2f}\,|\,\hat U(t_f,t_i)\,|\,x_{1i}\,x_{2i}\rangle
= \mathcal{N}\!\left(
\sum_{\substack{\text{TYPE }+1\ \text{paths}\\ i\to f}} e^{i S[\text{path}]/\hbar}
- \sum_{\substack{\text{TYPE }-1\ \text{paths}\\ i\to f}} e^{i S[\text{path}]/\hbar}
\right).
\end{equation*}

This case holds mathematically because it respects path composition. 


\subsection{Many Identical Particles}

The configuration space can be described as follows:
$$
\mathcal{C} = [(\mathbb{R}^D)^N - \Delta] / \sim 
,$$
where $\Delta$ represents \textbf{coincidences} where two particles are at the same position, and $\sim $ is the equivalence relation induced by the permutation of coordinates.

\subsubsection{2+1 D}

The paths in $2+1$ D can be classified using the braid group $B_N$. It is generated by $\sigma_1, \ldots , \sigma_{N-1}$, where $\sigma_i$  exchanges the $i$-th and the $(i+1)$-th strand while giving an overcrossing. 


A braid invariant is given by the \textbf{winding number}, which is the number of overcrossings minus the number of undercrossings. This is also the number of $\sigma_i$'s minus the number of $\sigma_i ^{-1}$'s in the braid word.

\subsubsection{3+1 D}

There is no knots in the 4 dimensional spacetime, so all the paths are simply classified by the symmetric group $S_N$. 


\subsubsection{Building a path integral}

Suppose the paths are classified by group $G$. We have
 $$
\langle \left\{ \vb x \right\} _f  | \hat{U} (t_f , t_i) | \left\{ \vb x \right\} _i  \rangle = \mathcal{N} \sum_{g \in  G} \rho(g) \sum_{\text{paths with type } g} e^{i S[\text{path}] / \hbar}
.$$

Here $\rho$ is a unitary representation of $G$. Adding $\rho(g)$ to the summation  is in line with our previous observation that adding a minus sign to TYPE -1 paths is a consistent move that respects path composition; and a different $\rho$ would  in fact correspond to a different type of particles. 









