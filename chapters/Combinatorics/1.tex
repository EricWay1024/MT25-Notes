\section{Power sets}


Let $[n] \vcentcolon= \left\{ 1,2, \ldots ,n \right\} $.
Let $\mathcal{P}(n)$ be the power set of $[n]$.
Let $[n]^{(k)}$ be the set of $k$-subsets of $[n]$; it is also called the $k$-th layer of $\mathcal{P}(n)$.

\begin{proposition}
  $| \mathcal{P}(n) | = 2^n$ and $| [n] ^{(k)} | = \binom{n}{k}$.  
\end{proposition}


$\mathcal{P}(n)$ is naturally a poset, where the relation $A \le B$  is given by set inclusion $A \subseteq B$. 

\begin{definition}
  The \textbf{symmetric difference} between two sets $A$ and $B$ is the set $A \bigtriangleup B \vcentcolon= (A \setminus B) \cup (B \setminus A)$. 
\end{definition}


If $| A \bigtriangleup B | = 1$, then either $A$ is obtained from removing an element from $B$, or $B$ is obtained from removing an element from $A$. 


\begin{definition}
  The \textbf{discrete cube} $Q_n$ is the graph with vertex set $\mathcal{P}(n)$ and an edge between $A$ and $B$ if and only if $| A \bigtriangleup B | = 1$.    
\end{definition}


We can also identify each element $A$ in $\mathcal{P}(n)$ with its \textbf{characteristic vector} in $\mathbb{F}^n_2$, which is $0$ resp. $1$ on the $i$-th position  if $i \in A$ resp. $i \not \in A$. 

Note that the symmetric difference corresponds to the XOR operation between two characteristic vectors.



\begin{proposition}
  $(A, B) \mapsto | A \bigtriangleup B |$ gives a metric on $Q_n$.   
\end{proposition}

\begin{proof}
  We only need to check the triangle inequality: $| A \bigtriangleup B| + |B \bigtriangleup C| \ge | A \bigtriangleup C |$. We can draw a Venn diagram for the three sets and partition $A \cup B \cup C$ into seven regions, and represent both sides by summing up their covered regions.  Then it becomes evident.
\end{proof}

\section{Antichains}

\begin{definition}
  A family $\mathcal{A} \subseteq  \mathcal{P}(n)$ is \textbf{chain} if all pairs 
\end{definition}

