% !TEX program = lualatex
\documentclass[11pt,a4paper]{book}

% ---------- Encoding & Language ----------
% \usepackage[utf8]{inputenc}
% \usepackage[T1]{fontenc}
% \usepackage{lmodern}      % switch from cmr → lmr
% \usepackage{cfr-lm}       % adds LF/OsF/TLF/TOsF shapes for Latin Modern
\usepackage[english]{babel}

% ---------- Page Layout ----------
\usepackage[margin=1in]{geometry}
\usepackage{fancyhdr}
\pagestyle{fancy}
\fancyhf{}
\fancyhead[L]{Notes on Differential Topology}
\fancyhead[R]{\leftmark}
\fancyfoot[C]{\thepage}

% ---------- Math Packages ----------
\usepackage{amsmath, amssymb, amsthm}
% \usepackage{mathabx}
\usepackage{mathtools}
\usepackage{physics} % for derivatives, bra-ket notation, etc.
\usepackage{bm}      % bold math symbols
\usepackage{csquotes}

% ---------- Extra Features ----------
\usepackage{graphicx}
\usepackage{tikz}       % for drawing diagrams
\usepackage{tikz-cd}
\usepackage{enumitem}   % for custom lists

\usepackage[colorlinks=true, linkcolor=blue]{hyperref}


\usepackage{array}
\usepackage{longtable}

% font
\usepackage{fontspec}
\usepackage{libertinus}
\setmainfont{Libertinus Serif}[Numbers={OldStyle,Proportional}]
\setsansfont{Libertinus Sans}
\setmonofont{Libertinus Mono}
\usepackage{unicode-math}
\setmathfont{Libertinus Math}
% \setmathfont{STIX Two Math} 

\usepackage[nameinlink,capitalize]{cleveref}

\usepackage[backend=biber,style=alphabetic]{biblatex}

\usepackage{changepage}   % for adjustwidth
\usepackage{environ}      % for capturing environment body


\addbibresource{ref.bib}  % <- points to the file above


% ---------- Theorem Environments ----------


% 1) 定义一个带缩进的定理样式
\newtheoremstyle{claimstyle}    % 样式名称
  {3pt}                       % 上方间距
  {3pt}                       % 下方间距
  {\normalfont}               % 正文字体
  {0pt}                       % 左侧缩进宽度——这里改成你想要的尺寸
  {\itshape}                 % 定理头字体
  {.}                         % 定理头后标点
  { }                         % 定理头后空隙
  {}                          % 定理头格式(留空表示自动“Claim <编号>”)

\theoremstyle{plain} % 斜体正文(默认)
\newtheorem{theorem}{Theorem}[section]
\newtheorem{lemma}[theorem]{Lemma}
\newtheorem{corollary}[theorem]{Corollary}
\newtheorem{proposition}[theorem]{Proposition}

\theoremstyle{definition} % 正常字体正文
\newtheorem{definition}[theorem]{Definition}
\newtheorem{example}[theorem]{Example}
\newtheorem{remark}[theorem]{Remark}

\theoremstyle{claimstyle}
\newtheorem{claim}{Claim}[theorem]   % “Claim” numbers inside each Theorem


% 2) define a new indented‐proof environment
\NewEnviron{indentedproof}[1][Proof]{%
  \begin{adjustwidth}{2em}{0pt}%  <-- change 2em to whatever indent you like
    \begin{proof}[#1]%
      \BODY
    \end{proof}%
  \end{adjustwidth}%
}
% ---------- Custom Commands ----------
\newcommand{\R}{\mathbb{R}}   % Real numbers
\newcommand{\N}{\mathbb{N}}   % Natural numbers
\newcommand{\Z}{\mathbb{Z}}   % Integers
\newcommand{\Q}{\mathbb{Q}}   % Rational numbers
\newcommand{\C}{\mathbb{C}}   % Complex numbers

% \newcommand{\mathbb{R}^n}{\mathbb{R}^n}
\newcommand{\Rm}{\mathbb{R}^m}
\newcommand{\Rk}{\mathbb{R}^k}
\newcommand{\Rl}{\mathbb{R}^l}
\newcommand{\Sn}{\mathbb{S}^n}
\newcommand{\RPn}{\mathbb{RP}^n}
\newcommand{\RP}{\mathbb{RP}}
\newcommand{\HH}{\mathbb{H}}
\newcommand{\Id}{\text{Id}}
\newcommand{\GL}{\text{GL}}
\DeclareMathOperator{\sign}{sign}
\DeclareMathOperator{\Diff}{Diff}
\newcommand{\supp}{\text{supp}\,}
\newcommand{\Rank}{\text{rank}\,}
\newcommand{\sbs}{\subseteq}
\newcommand{\TODO}{TODO}

% ---------- Document Begins ----------
\begin{document}

\title{Notes, Michaelmas Term 2025}
\author{Yuhang Wei}
\date{\today}
\maketitle
\tableofcontents
% \newpage


\chapter{Anyons and TQFT}


\subsection{Abstract formalism of quantum mechanics}

Since I have no physics background, I will try to build everything from first principles.

\begin{definition}
  A \textbf{Hilbert space} is a real or complex inner product space that is also a complete metric space with respect to the metric induced by the inner product.
\end{definition}


We will only consider complex Hilbert spaces in quantum mechanics. Given a Hilbert space $H$, let $P(H)$ be the projective Hilbert space of $H \setminus \left\{ 0 \right\} $ by modding out the equivalence relation $z \sim \lambda z$ for complex number $\lambda \neq 0$.     

\begin{definition}
  A \textbf{(pure) quantum state} is a \textit{ray} in $H$, i.e. an element in $P(H)$, which we identify as an element  in the unit ball of $H$. We denote it as a \textbf{ket} $| \psi \rangle$.  
\end{definition}



\begin{definition}
  A \textbf{bra} is an element in the dual space $H^*$, written as $\langle \phi |$. We write $\langle \phi | \psi  \rangle$  for $\langle \phi | (| \psi \rangle)$. This is just an inner product of two vectors $| \phi \rangle$ (which the dual of $\langle \phi |$) and $| \psi \rangle$.  
\end{definition}


Note that anything of the form $| n \rangle$ is a ket (thus a vector in $H$), even though (syntax-wise) $n$ might be a letter or a number. The same can be said for bras.


Fix a bra $\langle \phi |$ and a ket $| \psi \rangle$. We can then form an operator $\hat{O} = | \psi \rangle \langle \phi |$, which is a map $H \to H$. The Hermitian conjugate of this operator is $| \phi \rangle \langle \psi |$. In particular, we have $\hat{P}_{\psi} = |  \psi \rangle \langle \psi |$, which is a projection operator.  


Let $\left\{ | n \rangle \right\} $  be a complete basis of $H$ (note: this completeness is different from the completeness of $H$ as a metric space).  Then $\hat{I} = \sum_{ n} | n \rangle \langle n |$. 


The trace of an operation $\hat{K}$  is defined by $$
\operatorname{Tr} (\hat{K}) = \sum_{n } \langle n | \hat{K} | n \rangle
.$$

The trace is invariant under change of basis, and it has the cyclical property. 

An \textbf{observable} is a self-adjoint operator.

For any observable $\hat{O}$ and a ket $| \psi \rangle$, we have the expectation $$
\left< \hat{O} \right>_\psi \vcentcolon= \left< \psi | \hat{O} | \psi \right> = \operatorname{Tr} (\hat{O} \hat{P}_\psi)  = \operatorname{Tr} (\hat{P}_\psi \hat{O})
.$$ 







\section{Test test}


Hello world!!
\section{Particle Quantum Statistics}

\subsection{Single Particle Path Integral}


Let $\vb x (t) \in \mathbb{R}^D$ be the position of a single particle moving in $\mathbb{R}^D$ at time $t$. Let $\vb x_i$ be its initial position at time $t_i$ and $\vb x_f$ its final position at time $t_f$. Let $\hat{U}$ be a unitary time evolution operator, then the \textbf{propagator} gives the amplitude of starting at $\vb x_i$ and ending at $\vb x_f$, written as $$
\langle \vb x_f | \hat{U} (t_f, t_i) | \vb x_i \rangle
.$$        

Suppose we have an arbitrary wavefunction at $t_i$: $\psi (\vb x_i , t_i) = \langle \vb x_i | \psi (t_i)  \rangle$. Then we can use the propagator to propagate forward $\psi$ to $t_f$: $$
\langle \vb x_f | \psi (t_f) \rangle = \int d \vb x_i \langle \vb x_f | \hat{U} (t_f, t_i) | \vb x_i \rangle \langle \vb x_i | \psi (t_i)  \rangle
.$$  
In other words, the later wavefunction is a superposition of all ways the particle could have started at every $\vb x_i$, each weighted by the propagator from $\vb x_i$ to $\vb x_f$. 


The propagator has two properties:
\begin{itemize}
    \item It is unitary, so no amplitude is lost along the way.
    \item It respects path composition, so propagating from $t_i$ to $t_m$ and then from $t_m $ to $t_f$, is the same as propagating from $t_i$ to $t_f$. 
\end{itemize}

Feynman says that the propagator can be rewritten as 
$$
\langle \vb x_f | \hat{U} (t_f, t_i) | \vb x_i \rangle = \mathcal{N} \sum e^{i S[\vb x(t)] / \hbar }
,$$
where $\mathcal{N}$ is a constant,  the sum is over all paths $\vb x(t)$ (in the $(D+1)$-dimensional spacetime) from $(\vb x_i, t_i)$ to $(\vb x_f, t_f)$, and $S[\vb x(t)]$ is the classical action of the path (which is the integral of the Lagrangian).

For simplicity we will assume the Plunk constant $\hbar = 1$. 

The sum is dominant by the paths where the exponent oscillates the least, i.e. the paths with the least action. 

\subsection{Two Identical Particles}


If we have two identical particles, then the two states $| \vb x_1, \vb x_2 \rangle$  and $| \vb x_2, \vb x_1 \rangle$ are equivalent. We call the space of all states the \textbf{configuration space} $\mathcal{C}$. 



We call the non-exchange path TYPE $+1$ and the exchange path TYPE $-1$. Clearly their composition obeys the group operations of $\mathbb{Z}_2$. 


The path integral is now written as follows:

\begin{equation*}
\label{eq:3.4}
\langle x_{1f}\,x_{2f}\,|\,\hat U(t_f,t_i)\,|\,x_{1i}\,x_{2i}\rangle
= \mathcal{N}\sum_{\text{paths } i\to f} e^{i S[\text{path}]/\hbar}
= \mathcal{N}\!\left(
\sum_{\substack{\text{TYPE }+1\ \text{paths}\\ i\to f}} e^{i S[\text{path}]/\hbar}
+ \sum_{\substack{\text{TYPE }-1\ \text{paths}\\ i\to f}} e^{i S[\text{path}]/\hbar}
\right).
\end{equation*}

% This second line is simply a rewriting of the first having broken the sum into
% the two different classes of paths.


However, if we work with a different type of particles, we may have the following case:

\begin{equation*}
\label{eq:3.5}
\langle x_{1f}\,x_{2f}\,|\,\hat U(t_f,t_i)\,|\,x_{1i}\,x_{2i}\rangle
= \mathcal{N}\!\left(
\sum_{\substack{\text{TYPE }+1\ \text{paths}\\ i\to f}} e^{i S[\text{path}]/\hbar}
- \sum_{\substack{\text{TYPE }-1\ \text{paths}\\ i\to f}} e^{i S[\text{path}]/\hbar}
\right).
\end{equation*}

This case holds mathematically because it respects path composition. 


\subsection{Many Identical Particles}

The configuration space can be described as follows:
$$
\mathcal{C} = [(\mathbb{R}^D)^N - \Delta] / \sim 
,$$
where $\Delta$ represents \textbf{coincidences} where two particles are at the same position (or with a distance less than $\varepsilon$), and $\sim $ is the equivalence relation induced by the permutation of coordinates.

\subsubsection{2+1 D}

The paths in $2+1$ D can be classified using the braid group $B_N$. It is generated by $\sigma_1, \ldots , \sigma_{N-1}$, where $\sigma_i$  exchanges the $i$-th and the $(i+1)$-th strand while giving an overcrossing. 


A braid invariant is given by the \textbf{winding number}, which is the number of overcrossings minus the number of undercrossings. This is also the number of $\sigma_i$'s minus the number of $\sigma_i ^{-1}$'s in the braid word.

\subsubsection{3+1 D}

There is no knot in the 4 dimensional spacetime, so all the paths are simply classified by the symmetric group $S_N$. 


\subsubsection{Building a path integral}

Suppose the paths are classified by group $G$. We have
 $$
\langle \left\{ \vb x \right\} _f  | \hat{U} (t_f , t_i) | \left\{ \vb x \right\} _i  \rangle = \mathcal{N} \sum_{g \in  G} \rho(g) \sum_{\text{paths with type } g} e^{i S[\text{path}] / \hbar}
.$$

Here $\rho$ is a unitary representation of $G$. Adding $\rho(g)$ to the summation  is in line with our previous observation that adding a minus sign to TYPE -1 paths is a consistent move that respects path composition; and a different $\rho$ would  in fact correspond to a different type of particles. 


\subsection{Abelian Examples}


Now suppose $\rho$ is a one-dimensional representation of $G$; hence it is abelian. 

\subsubsection{3+1 D}

\begin{lemma}
  There are only two possible one-dimensional representations of $S_N$.  
\end{lemma}

\begin{proof}
    Clearly $\rho$ can only send a transposition to either $+1$ or $-1$.   
  Since $S_N$ is generated by transpositions, it suffices to show that if $\rho$ sends any one transposition to $+1$ resp.  $-1$, then $\rho$ sends all transpositions to $+1$ resp. $-1$. But we also know that any two transpositions are conjugate, and since $\rho$ is abelian, the result easily follows.      
\end{proof}

These two reps correspond to two different particles:

\begin{itemize}
    \item The trivial representation $\rho (g) = 1$ for all $g$. This corresponds to \textbf{bosons}.
    \item The alternating (sign) representation $\rho(g) = \operatorname{sgn} (g) = \pm 1$. This corresponds to \textbf{fermions}.    
\end{itemize}

\subsubsection{2+1 D}


We have a family of one-dimensional representations of $B_N$, parametrized by $\theta \in \mathbb{R} \pmod {2 \pi}$, given by $$
\rho(g) = e^{i \theta W(g)}
,$$
where $W(g)$ is the winding number of the braid $g$. In other words,
a clockwise resp. anticlockwise exchange accumulates a phase of $e^{i \theta}$ resp. $e^{- i \theta}$. 



\begin{itemize}
    \item If $\theta = 0$ then we recover \textbf{bosons}.
    \item If $\theta = \pi$ then we recover \textbf{fermions}.
    \item For any other value of $\theta$ we have \textbf{(abelian) anyons} or \textbf{fractional statistics}.   
\end{itemize}

\subsection{Non-abelian case}

\TODO







\section{Aharonov--Bohm Effect and Charge-Flux Composites}



\subsection{Aharonov--Bohm Effect}

I don't really know. But it seems that the effect says if there is a magnetic field between the two paths of the double-silt interference experiment, the amplitude (or the measured interference pattern) will be affected, even if the particle does not experience the (constant) field either directly or via the Faraday effect. Via some computation, what really happens is that the presence of this field gives a difference in accumulated phases the two paths. The phase difference is given by $$
\exp \left(  \frac{iq}{\hbar} \Phi_{\text{enclosed}}  \right)
,$$
where $\Phi_{\text{enclosed}}$ is the flux enclosed by the loop (path 1 - path 2). In other words, the phase shift is what we get when we move a charge (particle) around a flux.  


That seems to be all that matters.

\begin{itemize}
    \item If $\Phi \in \Phi_0 \mathbb{Z} $, where $\Phi_0 = 2 \pi \hbar / q$ is the elementary flux quantum, then the phase shift is in $2 \pi \mathbb{Z}$ and thus is equivalent to $0$.     
    \item We would get the same phase shift if we were to move flux around a charge. 
\end{itemize}


\subsection{Anyons as Charge-Flux Composites}


Now for some magical reasons, we model a particle as a charge-flux composite object, denoted as $(q, \Phi)$. 

\begin{proposition}
  If particle one $(q_1, \Phi_1)$  goes around the flux of particle two $(q_2, \Phi_2)$ , then the phase change is $e^{ i (q_1 \Phi_2 + q_2 \Phi_1) / \hbar}$. 
\end{proposition}

\begin{proof}
  This is because $q_1$ going around $\Phi_2$  gives a phase shift of $q_1 \Phi_2$, but at the same time $\Phi_1$ also goes around $q_2$ which gives $q_2 \Phi_1$.     
\end{proof}



\begin{proposition}
  If the two particles both have $(q, \Phi)$, then the phase for exchanging them is $e^{i q \Phi / \hbar}$.  
\end{proposition}


\begin{proof}
  This is because if we exchange them twice, that is equivalent to particle one going around particle two once.
\end{proof}


This indicates that these particles are $\theta$-anyons, where ${\theta} = q \Phi / \hbar$. 


\subsubsection{Spin of an anyon}

\begin{proposition}
  If we rotate the anyon around its axis, we gain a new phase of $e^{i \theta}$. 
\end{proposition}

\begin{proof}
  We can put the charge and the flux at slightly different positions; then the rotation is just moving the charge and the flux around each other, which gives us the result. 
\end{proof}

Note that the phase obtained by exchanging two identical particles is the same as the phase obtained by rotating one around its own axis. This matches the spin-statistics theorem. See Fig. 2.7 (p.10) of the book.


\subsubsection{Fusion of Anyons}

\begin{proposition}
  If we put two particles with $(q_1, \Phi_1)$ and $(q_1, \Phi_2)$   together, then we have a $(q_1+ q_2, \Phi_1+ \Phi_2)$ particle. 
\end{proposition}

\begin{proof}
  We can just pretend that we move far away enough so that they seem they are together.
\end{proof}



\subsubsection{Anti-Anyons}

The \textbf{anti-anyon} of the $(q, \Phi)$ anyon is just the particle with $(-q, -\Phi)$. If we fuse an anyon with its anti-anyon, we get the \textbf{vacuum particle}, $(0,0)$.


\subsection{Torus}

Now let us suppose our space is a torus (and as usual we have the one-dimensional time).
Denote the two nontrivial cycles around the torus as $C_1$ and $C_2$.   

Define two operations $T_1, T_2$ as follows. $T_i$ is the operator that creates a particle-antiparticle pair, moves the two in opposite directions  around the $C_1$ cycle until they meet on the opposite side of the torus and reannihilate. 

We also define $T_i ^{-1}$ as the time-reversed process of $T_i$.


\begin{proposition}
  Both $T_i$ commute with the Hamiltonian $H$; in other words, they preserve the energy. 
\end{proposition}
\begin{proof}
  They just do.
\end{proof}



\begin{proposition}
  $T_1$ does not commute with $T_2$.  
\end{proposition}

\begin{proof}
  It suffices to show that the operator $T_2  ^{-1} T_1 ^{-1} T_2 T_1$ is not trivial. See Fig. 4.9 of the book. The process is equivalent to one particle wrapping around another which gives a phase of $e^{-2 i \theta}$.   
\end{proof}


% \TODO{figure out what ground states mean}

\begin{proposition}
  If $| \alpha \rangle$ is an eigenstate of $T_1$ in the ground state space with $T_1 | \alpha \rangle = e^{i \alpha} | \alpha \rangle$ (which exists since $T_1$ commutes with Hamiltonian), then $T_2 | \alpha \rangle = | \alpha + 2 \theta \rangle$  is another ground state and is another eigenstate of $T_1$ with eigenvalue $e^{i (2 \theta + \alpha)}$.
\end{proposition}


\begin{proof}
  $T_2 |  \alpha \rangle$ must also be in the ground state space since $T_2$ commutes with the Hamiltonian.  Now $$
  T_1 (T_2 | \alpha \rangle) = e^{2 i \theta} T_2 T_1 |  \alpha \rangle = e^{i(2 \theta + \alpha)} (T_2 |  \alpha \rangle)
  .$$
  Therefore $T_2 |  \alpha \rangle$ is also an eigenstate of $T_1$, which must also be in the ground state space.   
\end{proof}

\begin{corollary}
  If $\theta = \pi p / m$ for coprime integers $p, m$, then there are $m$ independent ground states.
\end{corollary}


\begin{proof}
  This is easy, since we can apply the above procedure $m$ times before we get the phase $\alpha + 2 \pi$ which is equivalent to $\alpha$.   
\end{proof}





\chapter{Combinatorics}
\section{Power sets}


Let $[n] \vcentcolon= \left\{ 1,2, \ldots ,n \right\} $.
Let $\mathcal{P}(n)$ be the power set of $[n]$.
Let $[n]^{(k)}$ be the set of $k$-subsets of $[n]$; it is also called the $k$-th layer of $\mathcal{P}(n)$.

\begin{proposition}
  $| \mathcal{P}(n) | = 2^n$ and $| [n] ^{(k)} | = \binom{n}{k}$.  
\end{proposition}


$\mathcal{P}(n)$ is naturally a poset, where the relation $A \le B$  is given by set inclusion $A \subseteq B$. 

\begin{definition}
  The \textbf{symmetric difference} between two sets $A$ and $B$ is the set $A \bigtriangleup B \vcentcolon= (A \setminus B) \cup (B \setminus A)$. 
\end{definition}


If $| A \bigtriangleup B | = 1$, then either $A$ is obtained from removing an element from $B$, or $B$ is obtained from removing an element from $A$. 


\begin{definition}
  The \textbf{discrete cube} $Q_n$ is the graph with vertex set $\mathcal{P}(n)$ and an edge between $A$ and $B$ if and only if $| A \bigtriangleup B | = 1$.    
\end{definition}


We can also identify each element $A$ in $\mathcal{P}(n)$ with its \textbf{characteristic vector} in $\mathbb{F}^n_2$, which is $0$ resp. $1$ on the $i$-th position  if $i \in A$ resp. $i \not \in A$. 

Note that the symmetric difference corresponds to the XOR operation between two characteristic vectors.



\begin{proposition}
  $(A, B) \mapsto | A \bigtriangleup B |$ gives a metric on $Q_n$.   
\end{proposition}

\begin{proof}
  We only need to check the triangle inequality: $| A \bigtriangleup B| + |B \bigtriangleup C| \ge | A \bigtriangleup C |$. We can draw a Venn diagram for the three sets and partition $A \cup B \cup C$ into seven regions, and represent both sides by summing up their covered regions.  Then it becomes evident.
\end{proof}

\section{Chains and antichains}

\begin{definition}
  Let $P$ be a poset. A \textbf{chain} resp. \textbf{antichain} in $P$ is a subset $\mathcal{A}$ of $P$ such that $A$ and $B$ are comparable resp. incomparable for any distinct $A, B \in  \mathcal{A}$.   A chain (resp. antichain) $\mathcal{A}$  is \textbf{maximal} if no other element of $P$  can be added to $\mathcal{A}$ so that $\mathcal{A}$ remains a chain (resp. antichain). 
\end{definition}

Note that if we say the largest (anti)chain, that means the (anti)chain with the largest size.

The following is evident from the definition.
\begin{lemma} \label{lem:chain-antichain-at-most-one-common}
  A chain and an antichain have at most one common element.
\end{lemma}

We first study the chains and antichains in $\mathcal{P}(n)$. 

\begin{proposition}
  Every maximal chain in $\mathcal{P}(n)$ has $(n+1)$ elements, and there are $n!$ of them.   
\end{proposition}

\begin{proof}
  Each of a maximal chain can be identified as the process of adding all the $n$ numbers to the empty set one by one. 
\end{proof}


A more interesting result is the following.

\begin{theorem}
  The largest antichain in $\mathcal{P}(n)$ has size $\binom{n}{\lfloor \frac{n}{2} \rfloor}$.  
\end{theorem}


Clearly each layer of $\mathcal{P}(n)$ is an antichain and among them the largest size is  $\binom{n}{\lfloor \frac{n}{2} \rfloor}$. We have to show this is the largest possible size for an antichain. This is stated as:

\begin{theorem}[Sperner's Lemma]
  An antichain in $\mathcal{P}(n)$ has size at most $\binom{n}{\lfloor \frac{n}{2} \rfloor}$, which is only attained when it is a middle layer.
\end{theorem}

(By a middle layer we want to gloss over the cases when $n$ is odd or even.)


\subsection{Sperner's Lemma: first proof}

By \cref{lem:chain-antichain-at-most-one-common}, we cannot partition $\mathcal{P}(n)$ into fewer than $\binom{n}{\lfloor \frac{n}{2} \rfloor}$ chains (since we have an antichain of that size). The question is whether this is an attainable minimum. The answer is yes, and it implies Sperner's Lemma (albeit without giving the condition for attaining the extremum).

\begin{lemma}[Hall's Theorem]
  Let $G = (V, E)$ be a bipartite graph with vertex classes $X$ and $Y$. Then $G$ has a complete matching from $X$ to $Y$ if and only if for all $S \subseteq  X$ we have $\Gamma(S) \ge S$ (Hall's condition), where $\Gamma(S)$ is the \textbf{neighborhood} of $S$ defined as the set of vertices $u$ such that there is an edge connecting $u$ and some $v \in S$.  
\end{lemma}

A \textbf{complete matching} from $X$ to $Y$ is defined as an edge set $M = \left\{ e_x \right\} _{x \in X}$ such that any $y \in Y$ is incident to at most one edge in $M$.  If the edges are thought of as a way of mapping then this gives an injective map from $X$ to $Y$. 


\begin{lemma}
  There is a partition of $\mathcal{P}(n)$ into  $\binom{n}{\lfloor \frac{n}{2} \rfloor}$ chains. 
\end{lemma}

\begin{proof}
  In $Q_n$, we claim that for $r < \frac{n}{2}$, there is a complete matching from $[n]^{(r)}$ to $[n]^{(r+1)}$.
  
  We consider the bipartite subgraph $G$ of $Q_n$ whose vertex set consists of these two layers. By Hall's Lemma, we only have to verify Hall's condition. The trick is a double counting argument. Let $S \subseteq [n]^{(r)}$ and $T = \Gamma(S)$. We double count the number of edges $e(S, T)$ between $S$ and $T$. 

  Each $A \in S$ has degree $(n-r)$ since there are $(n-r)$ ways to add an element to $A$ to make an $(r+1)$-set. Therefore $e(S, T) = (n-r) |S|$. 
  
  In the graph $G$, each  $B \in [n]^{(r+1)}$ has degree $(r+1)$. However, if $B \in T$ and we remove an element from $B$, we are not guaranteed to get a set in $S$. (Simple example: $n = 2$, $r=1$,   $S = \left\{ \left\{ 1 \right\}  \right\} $ and $T = \left\{ \left\{ 1,2 \right\}  \right\} $, but removing $1$ from $\left\{ 1,2 \right\} $ does not give us something in $S$.) Therefore $e(S, T) \le (r+1) |T|$. 
  
  Then some calculation reveals that $|T| \ge |S|$ as $r < \frac{n}{2}$. Thus Hall's condition is satisfied and our claim is proven. 
  
  By symmetry there is a complete matching from $[n]^{(r)}$ to $[n]^{(r-1)}$ if $r >\frac{n}{2}$. We can then glue these matchings together and we get the desired partition. (Imagine ``collapsing'' the layers from both ends towards the middle.)  
\end{proof}


\begin{proof}[Sperner's Lemma, first proof]
  By \cref{lem:chain-antichain-at-most-one-common}, we see that no antichain can contain more than $\binom{n}{\lfloor \frac{n}{2} \rfloor}$ elements.
\end{proof}

\subsection{Sperner's Lemma: second proof; LYM inequality}


\begin{theorem}[LYM Inequality]
  Let $\mathcal{F} \subseteq  \mathcal{P}(n)$ be an antichain. Then $$
  \sum_{i = 0}^{n} \frac{|\mathcal{F} \cap [n]^{(i)}|}{\binom{n}{i}} \le 1
  .$$ 
  The equality is attained if and only if $\mathcal{F} = [n]^{(i)}$ for some $i$. 
\end{theorem}


The theorem says that it is more ``expensive'' to take a set away from the middle layer since then the ``weight'' $\frac{1}{\binom{n}{i}}$ is larger. An extreme case: if we take the empty set, then it contributes $1$ to the summand and we are forced to take no other sets.



\begin{proof}[Sperner's Lemma, second proof, assuming LYM Inequality]
  Let $\mathcal{F} \subseteq \mathcal{P}(n)$ be an antichain. Then we have $$
  1 \ge \sum_{i = 0}^{n} \frac{|\mathcal{F} \cap [n]^{(i)}|}{\binom{n}{i}}  \ge \sum_{i = 0}^{n} \frac{|\mathcal{F} \cap [n]^{(i)}|}{\binom{n}{\lfloor n / 2 \rfloor}} = \frac{|\mathcal{F}|}{\binom{n}{\lfloor n / 2 \rfloor}}
  ,$$ 
  since $\binom{n}{\lfloor n / 2 \rfloor} \ge \binom{n}{i}$  for all $i$. 
  If we have equality, then $\mathcal{F} = [n]^{(i)}$ and $i = \lfloor \frac{n}{2} \rfloor$ or   $i = \lceil   \frac{n}{2} \rceil$.
\end{proof}



\subsection{LYM inequality: first proof; Local LYM}

Let $\mathcal{F} \subseteq [n]^{(k)}$. The lower shadow $\partial \mathcal{F}$ of $\mathcal{F}$ is $\left\{ B \in [n]^{(k-1)} : B \subseteq A \text{ for some } A \in \mathcal{F} \right\} $.


This is just the neighborhood of $\mathcal{F}$ in the graph $Q_n$ intersecting the lower layer. 


\begin{lemma}
  Let $\mathcal{A} \subseteq [n]^{(r)}$. Then $$
  \frac{| \partial \mathcal{A} |}{\binom{n}{r-1}} \ge \frac{| \mathcal{A} |}{\binom{n}{r}}
  .$$
  Equality is attained if and only if $A$ is $\varnothing$ or $[n]^{(r)}$.  
\end{lemma}

\begin{proof}
  Also a double counting argument, much akin to what we have seen. See notes.
\end{proof}




\chapter{Surfaces and 3-Manifolds}
\section{Moduli spaces}


\begin{definition}
  A \textbf{hyperbolic surface} is a $2$-dimensional manifold equipped with a Riemannian metric with curvature $-1$.  
\end{definition}


\begin{definition}
  A \textbf{Riemannian surface} is a $1$-dimensional complex manifold, where the transition maps between charts are holomorphic. 
\end{definition}

\begin{definition}
  $$\operatorname{PSL}_2 \mathbb{R} \vcentcolon= \left\{ \begin{pmatrix}
    a & c \\
    b & d 
  \end{pmatrix} : ad - bc \neq 0  \right\} / \left\{ \begin{pmatrix}
  \lambda & 0 \\
  0 & \lambda
  \end{pmatrix} : \lambda \in \mathbb{R}_{\neq 0} \right\}.  $$ 
  A discrete subgroup of $\operatorname{PSL}_2 \mathbb{R}$ is \TODO discrete topology?
\end{definition}

\begin{theorem}
  Fix $g \ge 2$. The following sets are equivalent:
  \begin{enumerate}
    \item Hyperbolic surfaces of genus $g$;
    \item Riemannian surfaces of genus $g$;
    \item Discrete subgroups $\Gamma \le \operatorname{PSL}_2 \mathbb{R}$ such that $\Gamma \cong \pi_1 \Sigma_g$ (up to conjugation).  
  \end{enumerate} 
\end{theorem}


\begin{proof}
  $1 \implies 3$. Take a hyperbolic surface $\Sigma_g$  with Riemannian metric $d$. Take the universal covering $\tilde{\Sigma}_g$, and lifting $d$ gives a metric $\tilde{d}$ on $\tilde{\Sigma}_g$. Since the covering space will also have curvature $-1$, but a simply-connected space with curvature $-1$ is uniquely the hyperbolic plane, we have an identification $\tilde{\Sigma}_g = \mathbb{H}^2$. \TODO{two models of the hyperbolic plane.} Then since $\pi_1 \Sigma_g$ acts (proper discontinuously) on the covering space $\tilde{\Sigma}_g$, we have that $\pi_1 \Sigma_g$ acts on $(\tilde{\Sigma} _g, \tilde{ds^2})$ by orientation-preserving isometries, i.e. $\pi_1 \Sigma_g \to \operatorname{Isom}^+ (\mathbb{H}^2)$. But this is equivalent to Mobius transformations with discrete image and thus to discrete subgroups of $\operatorname{PSL}_2 \mathbb{R}$.  (The hyperbolic plane does not have a fixed basepoint, so the subgroups are up to conjugation).

  $3 \implies 1$. We simply take the quotient space $\mathbb{H}^2 / \Gamma$. One can check that this indeed has the right genus.
  
  $2 \implies 3$. Suppose we have $(\Sigma_g, J)$, where $J$ is an atlas of charts. Consider the universal covering $\tilde{\Sigma}_g$, and $J$ induces an atlas $\tilde{J}$. Now a simply-connected Riemannian surface  can only be a Riemann sphere, a complex plane or a disk. However the first two cannot cover $\Sigma_g$ \TODO, and therefore $\tilde{\Sigma}_g$ must be a disk, i.e. $(\tilde{\Sigma}_g, \tilde{J}) \cong D^2$ as Riemannian surfaces. Now $\pi_1 \Sigma_g$ acts on $\tilde{\Sigma}_g$, so we have a map $\pi_1 \Sigma_g \to \operatorname{Aut}(D^2) = \operatorname{PSL}_2 \mathbb{R}$ (Mobius transformations again). We have discrete image due  to the proper discontinuity of the action.   

  $3 \implies 2$. Still take a quotient $D^2 / \Gamma$.  
\end{proof}

\begin{example}
    Tile $\mathbb{H}^2$ with octagons. Let $\Gamma$ be the group of isometries which preserve the tiling. Then $\mathbb{H}^2 / \Gamma$ is equivalent to an octagon with side identifications which is exactly the genus-2 surface.     
\end{example}


\begin{definition}
  The set above for $g \ge 2$ is called the \textbf{moduli space} $M_g$.   
\end{definition}


We now attempt to equip such set with a metric, called the \textbf{Teichmuller metric}, to make it a metric space. Let $X_1, X_2$ be two Riemann surfaces.  Let $f : X_1\to X_2$ be an orientation-preserving diffeomorphism. First, define for any $p \in X_1$  $$
\mu_f (p) = \frac{\partial_{\bar{z}} f}{\partial_z f} |_p
.$$ Then define $$
K (f) = \sup_{p \in X_1} \frac{1 + | \mu_f (p)| }{1 - | \mu_f (p) | }
.$$ 
Then $$
d_{\text{Teich}}(X_1, X_2) = \log \inf \left\{ K(f) : \text{orientation-perserving diffeomorphism } f : X_1\to X_2\right\} 
.$$ 


To see what this all means, first take the derivative $Df : TX_1 \to TX_2$. Then as $f$ sends $p \in X_1$ to $f(p) \in X_2$, $Df$ sends $v \in TX_1$ to $Df(v)$ and $iv \in TX_1$ (which is perpendicular to $v$) to $Df(iv)$. $f$ is \textbf{perfectly conformal} if $Df(i v) = i Df(v)$, i.e. the perpendicularity is preserved; but this is not always the case. In general, $Df = \partial_z f + \partial_{\bar{z}} f$, where the second term vanishes if $f$ is perfectly conformal; in which case $\mu_f = 0$. 

For $K(f)$, we find the point where the distortion is the worst. If we take an infinitesimal circle around $p \in X_1$, then $f$ will send this circle to some ellipse with long axis $a$ and short axis $b$. Then $K(f) = \sup_{p \in X_1} \frac{a}{b}$.  Therefore, $K(f) = 1$ only for perfectly conformal maps; otherwise, $K(f) > 1$. 

Then $d_{\text{Teich}}$ is a measurement of how close the ``best'' diffeomorphism $f$ between $X_1$ and $X_2$ is to preserving the Riemannian structure. 


In this way we make $M_g$ a metric space; and thus a topological space. 


\begin{proposition}
  $(M_g, d_{\text{Teich}})$ is a $(6g-6)$-dimensional manifold; it is not simply-connected.   
\end{proposition}

\begin{proposition}
  $(M_g, d_{\text{Teich}})$ is not Riemannian but Finslerian.  
\end{proposition}


Let $\Sigma$ be an orientable smooth surface. Define $\operatorname{Teich}(\Sigma)$ as the space of hyperbolic metrics on $\Sigma$, equivalently complex structures on $\Sigma$, equivalently discrete and faithful homomorphisms $\pi_1(\Sigma) \to \operatorname{PSL}_2 \mathbb{R} / \text{conjugation}$.     
Then the Teichmuller space is topologically equivalent to $\mathbb{R}^{6g-6}$ and is   a covering space of $M_g$. 




\newpage
\printbibliography
\end{document}





